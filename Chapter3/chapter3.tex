%!TEX root = ../thesis.tex
%*******************************************************************************
%****************************** Third Chapter **********************************
%*******************************************************************************
\chapter{Optimal Planning with Pattern Databases}

% **************************** Define Graphics Path **************************
\ifpdf
    \graphicspath{{Chapter3/Figs/Raster/}{Chapter3/Figs/PDF/}{Chapter3/Figs/}}
\else
    \graphicspath{{Chapter3/Figs/Vector/}{Chapter3/Figs/}}
\fi

\section{Introduction}

The automated generation of search heuristics is one of the holy grails in AI, and goes back to early work of Gaschnik ~\cite{Gaschnik:79}, Pearl ~\cite{Pearl:heuristics}, and Prieditis ~\cite{Preditis:Discovery}. In most cases, lower bound heuristics are problem relaxations: each plan in the original state space maps to a shorter one in some corresponding abstract one. In the worst case, searching the abstract state spaces at every given search nodes exceeds the time of blindly searching the concrete search space ~\cite{Valtorta:84}. With pattern databases (PDBs), all efforts in searching the abstract state space are spent prior to the plan search, so that these computations amortize through multiple lookups. 

Initial results of Culberson and Schaeffer ~\cite{Culberson:pdb-first} in sliding-tile puzzles, where the concept of a pattern is a selection of tiles, quickly carried over to a number of combinatorial search domains, and helped to optimally solve random instances of the Rubik's cube, with non-pattern labels being removed ~\cite{Korf:Rubik}. When shifting from breadth-first to shortest-path search, the exploration of the abstract state-space can be extended to include action costs.

The combination of several databases into one, however, is tricky ~\cite{Haslum:07}. While the maximum of two PDBs always yields a lower bound, the sum usually does not. Korf and Felner ~\cite{Korf:DisjointPDB} showed that with a certain selection of disjoint (or additive) patterns, the values in different PDBs can be added while preserving admissibility. Holte et al. ~\cite{Holte:MultiplePDb} indicated that several smaller PDBs may outperform one large PDB. The notion of a pattern has been generalized to production systems in vector notation ~\cite{Holte:99}, while the automated pattern selection process for the construction of PDBs goes back to the work of Edelkamp ~\cite{Edelkamp:GA}. 

Many planning problems can be translated into state spaces of finite domain variables ~\cite{Helmert:MultiVariate}, where a selection of variables (pattern) influences both states and operators. For disjoint patterns, an operator must distribute its original cost, if present in several abstractions ~\cite{Katz:2008Optimal,Yang:08General}. 

During the PDB construction process, the memory demands of the abstract state space sizes may exceed the available resources. To handle large memory requirements, symbolic PDBs succinctly represent state sets as binary decision diagrams ~\cite{Edelkamp:SymbolicPlan}. However, there are an exponential number of patterns, not counting alternative abstraction and cost partitioning methods. Hence, the automated construction of informative PDB heuristics remains a combinatorial challenge. Hill-climbing strategies have been proposed ~\cite{Haslum:07}, as well as more general optimization schemes such as genetic algorithms ~\cite{Edelkamp:GA,Franco:CPC}. The biggest area of research in this area remains the quality evaluation of a PDB (in terms of the heuristic values for the concrete state space) which can only be estimated. Usually, this involves generating the PDBs and evaluating them ~\cite{Edelkamp:PlanningPDb,Korf:Rubik}. 

\section{Automatically Created Pattern Databases}

Planning is a PSPACE-complete problem \cite{Bylander:Complexity}, heuristic search has proven to be one of the best ways to find solutions in a timely manner. 

\subsection{Pattern Selection}

\section{Results}